\documentclass[11pt, a4paper]{article}
\usepackage{amsfonts, amsmath, hanging, hyperref, parskip, times}
\usepackage[numbers]{natbib}
\usepackage[pdftex]{graphicx}
\hypersetup{
  colorlinks,
  linkcolor=blue,
  urlcolor=blue,
  citecolor=blue
}

\let\section=\subsubsection
\newcommand{\pkg}[1]{{\normalfont\fontseries{b}\selectfont #1}} 
\let\proglang=\textit
\let\code=\texttt 
\renewcommand{\title}[1]{\begin{center}{\bf \LARGE #1}\end{center}}
\newcommand{\affiliations}{\footnotesize\centering}
\newcommand{\keywords}{\paragraph{Keywords:}}

\setlength{\topmargin}{-15mm}
\setlength{\oddsidemargin}{-2mm}
\setlength{\textwidth}{165mm}
\setlength{\textheight}{250mm}

\begin{document}
\pagestyle{empty}

\title{DocOpt, super easy command line options in \proglang{R}}

\begin{center}
  {\bf Edwin de Jonge$^{1,^\star}$}
\end{center}

\begin{affiliations}
1. Statistics Netherlands (CBS) \\[-2pt]
$^\star$Contact author: \href{mailto:e.dejonge@cbs.nl}{e.dejonge@cbs.nl}\\
\end{affiliations}

\keywords Options, Rscript, docopt

\vskip 0.8cm

With its increasing popularity \proglang{R} scripts are more and more executed in batch mode from the command line. 
When a script matures and becomes more generic it often is desirable to add command line options to 
your script. Starting simple you may use \code{cmdArgs} to parse the extra options given to the script, but it quickly 
becomes complicated. The packages \pkg{getopt}~\cite{getopt} and \pkg{optparse}~\cite{optparse} can be of great help for parsing the options, but it \pkg{docopt} makes it even easier.

The package \pkg{docopt} is a R port of the \proglang{Python} package \code{docopt}.
Docopt is a command-line interface description language, and help you to formulate a 
command-line interface and automatically generate a parser for it. 
The definition of the command line interface is also its documentation.




%% The \proglang, \code, and \pkg macros may be reused


%% references: 
\nocite{docopt,van2007python}
\bibliographystyle{chicago}
\bibliography{docopt}

%% references can alternatively be entered by hand
%\subsubsection*{References}

%\begin{hangparas}{.25in}{1}
%AuthorA (2007). Title of a web resource, \url{http://url/of/resource/}.

%AuthorC (2008a). Article example in proceedings. In \textit{useR! 2008, The R
%User Conference, (Dortmund, Germany)}, pp. 31--37.

%AuthorC (2008b). Title of an article. \textit{Journal name 6}, 13--17.
%\end{hangparas}

\end{document}
